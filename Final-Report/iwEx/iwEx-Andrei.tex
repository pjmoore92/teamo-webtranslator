\documentclass{article}
\usepackage{marginnote}

\title{Website for a translator\\ \vspace{4mm} 
Individual writing exercise}

\author{\bf Andrei Mustata\\ \bf Team O}

\date{\today}

\begin{document}
\maketitle

The project is to build a website for a translator (the human languages kind).

\section*{Background}

Currently, the translator has her usual clients from whom she receives documents
intended to be translated. They agree on a price and then she starts working
on it. After the document is translated, she sends it back to them and waits
for her payment. All communication is done mainly by e-mail.

While this might look like a nice and comfortable process, issues might occur.
For instance, having the document translated and in their hands, clients may
not pay at all. Of course, this would give that particular client a bad
reputation and this may not be a frequent situation, but it is certainly not
a happy case. Another problem rises when the number of the clients gets too
big. Managing only a handful of documents from a couple of clients is fine
to do without a software system, but when the demand increases, things are
bound to go wrong. Also, communication becomes difficult and hard to track.


\section*{Aims}

The purpose of our project is to solve the issues discussed earlier.

A few important points on why using a software system to manage documents,
keep track of discussions and tasks would be better than the current system
are:
\marginpar{\textbf{Karen}: \emph{Also shows a timeline of her activity.}}
\begin{itemize}
	\item
	makes {\bf planning} deadlines easier, since computers are better than humans
	at taking large amounts of data and comparing, adding, etc.;
	\item
	makes {\bf file managing} easier;
	\item
	improves {\bf productivity} - now that the other two are out of the way, the
	user of the system has more time to think of things that matter more.
\end{itemize}

By having all the details clearly stated and planned, the clients are encouraged
to pay more attention and to take it more seriously. It also makes both sides
- client and translator - more accountable.

Having a detailed log of all the documents delivered to which clients and when,
would make keeping track of the activity easier to do.

By using an electronic payment system, the risk of having unpaid jobs is greatly
reduced. It also gives the business a more professional look.
\marginpar{\textbf{Karen}: \emph{Using Paypal makes her less vulnerable to hacking
attacks.}}

\section*{Motivation}
First of all, the final product is to be used in a real business, so it gives
us a chance to see how it's like to work with a real client for a change.
The difference between real-life clients and, say, Computing Science lecturers
is that clients are usually not technical people. It's highly unlikely that a 
client would know exactly what they want from the very beginning or having
fixed requirements that never change, not to mention the possibility of them
having a thorough description of their project so we could get started on
the coding. This gives us a good experience on interacting with other people,
therefore improving our skills to explain computer-related concepts to people
that work in other fields and also to convert their natural language
requirements into technical ones.


Secondly, it's a project that \emph{lives} on the Internet. Ten years ago, the
Internet was everywhere, but now it's even more so. It's a rapidly evolving
area where new technologies and new techniques are developed very quickly.
These are also easy to use by anyone, regardless of computer knowledge,
nature of their device (mobile or desktop) or their operating system. All they
need is an up-to-date web browser, a requirement that is easy to fulfill,
since most of them update automatically. So, there is a large audience that
is easy to get to.

Another reason for choosing a web-based system is the fact that it is
relatively easy to deploy a project. There are a range of web hosting services
available that are also cheap to rent: starting from around £20 per year for 
small, basic accounts on a shared server and going up to over a few hundred 
pounds for more complex and specialized accounts. Of course, the services and 
support provided vary greatly, so having both good and bad experiences with 
a few of these providers was essential and it helped us narrow the list down
to a few. Since this is our client's first website, we advised that a simple,
cheaper hosting plan, but with an adequate amount of storage space should suffice
for now. Also, out of the abundance of technologies which we could have used,
we chose to use the {\bf Linux}, {\bf Apache}, {\bf MySQL} and {\bf PHP} (LAMP) stack, for a few reasons:
\begin{itemize}
	\item
	it is very {\bf popular}, so help was always easy to find;
	
	\item
	it is {\bf easy} to {\bf learn}, {\bf use} and {\bf setup}.
\end{itemize}
Also, we thought that this choice would lower the costs of an eventual upgrade
of the system, as opposed to having used some more exotic technologies that
come at the cost of losing the robustness and stability that we needed.
\marginpar{\textbf{Karen}: \emph{* good point.}}

\section*{Outline}
In the following chapters, we will go over the processes that we went through
to deliver the final product.

In the {\bf Specification and Design} chapter we will talk about the functional and
non-functional requirements capturing process - we started with a small set of
initial requirements, then we had interviews with the client, which lead to a
further understanding of the problem and the discovery of more requirements.
Also, the system architecture and user interface that we planned are discussed
here. All of the concepts and diagrams presented here should have been thoroughly
explained for everyone, 

The {\bf Implementation} section also describes the system as well as the difficulties that we encountered, but at a finer level of detail.

The {\bf Testing} chapter is concerned with all the tests that we put our product through in order to check if our solution met the requirements (e.g.: usability testing, acceptance testing).

The final section of the report will list the references to the work of others that we have cited throughout the document.

\end{document}
