% This example An LaTeX document showing how to use the l3proj class to
% write your report. Use pdflatex and bibtex to process the file, creating 
% a PDF file as output (there is no need to use dvips when using pdflatex).

% Modified 

\documentclass{l3proj}
\usepackage{wrapfig}
\usepackage{url}
\usepackage{appendix}
\begin{document}
\title{A Website Translation Service}
\author{Alasdair Campbell \\
	Andrei Mustata \\
	Paul Moore \\
        Stephen Hayton \\
	Wei Zhang}
\date{28th February 2012}
\maketitle
\begin{abstract}

Since the writing of the earliest manuscripts and documents there has been a need for translation so that the people of the world could understand what had been written or said, in their own native language. Translation is just as relevant today as it was thousands of years ago, although the processes have modernised considerably. The focus of this project is to create an online presence and document delivery system for a Glasgow based translator who is creating her own document translation business. Our goal is to improve upon the translation sites currently available and to give a unique, modern and fresh feel to translation.

\end{abstract}
\educationalconsent
\tableofcontents
%==============================================================================
\chapter{Introduction}
\label{intro}
\textit{\small{This article assumes a basic general computing knowledge from the reader. If any words or phrases are not understood, consult the glossary of terms (Section~\ref{sec:gloss}) at the end of the document. All content is owned by the authors of this document or is otherwise referenced in bibliography (Section~\ref{sec:bibl})}}\\
\section{Background}
As part of our degree in third year we are tasked with a team project. This relates to computing disciplines new to us and draws on knowledge gained from the preceding two years at University. Teams were randomly assembled at the beginning of semester and our team received the project to create a website providing a document translation service for a real client. It was a pleasing allocation mainly due to the latter part of the task: the fact we would be working with a real client. This bespoke web-based service builds on our client's existing business model, with the specific aims to bring about an expansion in her company, while also facilitating an improvement to her working practice as a whole.\\
\\
There are many translation services available on the Internet already, a simple Google search for online translation returns around one hundred and fifty nine million results. We have looked into various different types of translation and interpretation websites during our research and have found pros and cons from each. This vast number of already available websites creates a desire of competing with what's already available, by trying to improve areas where other websites have fallen short. We believe one of the key factors that makes a modern website successful is minimalistic design: it offers simple and effective functionality and is aesthetically pleasing. The combination of these things means users are more likely to use the website after stumbling across it in a search, perhaps, and will ultimately give our client a larger customer base. To clarify, we are not trying to re-invent the wheel with our project. We have used frameworks and other free source components in the development of our website. The system is mainly built around the LAMP structure, i.e., Linux + Apache + MySQL + PHP. We believed that these choices would lower the costs of an eventual upgrade of the system, as opposed to having used some more exotic technologies that come at the cost of losing the robustness and stability that we needed. We aimed to maintain a user friendly feel and look and believe that is something we accomplished well. One of the things that encourages this notion is the fact we have created a simple 3-stage process in which users can register, upload documents, and request languages to translate to.\\ 
\section{Aims}
To briefly summarise our task at hand: we are to develop a website for a free-lance translator. It should allow customers to upload documents, request one of the available languages to translate to, and submit a request for translation. The translator should then be able to review those submitted documents and send a quote to the customer for the job to be translated. If happy, the customer should then be able to pay for the job(s) via Paypal, and then receive their translated document after a specified period of time. Additional required features of the website will be discussed later. Listed below are a few important points on why using a software system to manage documents and to keep track of discussions and tasks has advantages over the current manual system:

\begin{itemize}
	\item
	It makes {\bf planning} deadlines easier, since computers are better than humans
	at taking large amounts of data and comparing, adding, etc.;
	\item
	It makes {\bf file management} easier;
	\item
	It improves {\bf productivity} - due to the two above points, the translator now has an organised system and has more time to spend doing the actual translation.
\end{itemize}

By having all the job details clearly stated and planned, the customers are encouraged to pay more attention and to take it more seriously. It also introduces the notion of accountability on both sides - customer and translator. Having a detailed log of all the documents delivered to customers makes keeping track of all the activities easier to do. By using an electronic payment system, the risk of having unpaid jobs is eliminated. It also gives the business a more professional look and makes our client less vulnerable to hacking attacks as opposed to other methods of payment. As part of the specification requirements, our website is required to support Paypal for payment of translated document. To provide a short brief of this service: "PayPal allows any business or consumer with an email address to securely, conveniently and cost-effectively send and receive payments online " - \textit{\small{www.paypal.com/about}}. This setup provides the translator with a feature that will not only provide her with security but also will encourage her customers to use the site, since Paypal has a massive online presence with excellent reputation. It is seen as the medium of paying for goods online, and this helps contribute to a modern, reliable website that we can develop. The translator, therefore, does not need to be concerned with things like credit card verification, since this is all functionality covered by Paypal. This maintains a level of separation between herself and her customers, so that she absolutely knows if her work has been paid for before she commits to it. \\

\section{Motivation}
An essential part of our requirements gathering process was the first meeting with the translator, Joelle Cimatche. Our team had been forewarned by our supervisor that the client was not very "technically minded", so we tried our best to prepare questions that did not assume she had much experience with a computer. It's fair to say that after reading the initial brief for the project we made assumptions as to the needs of the client, addressing the problem from our viewpoint. Realising we should expect to review these assumptions during the process of working on the requirements capture is a valuable lesson for future employment. Working with an individual or organization, firmly on the periphery of Internet technology adds it's own challenges, particularly during the requirements capturing process. At the meeting, the client explained she had a very basic understanding of word processing applications and the Internet, and not much else. This presented us with an additional challenge. We couldn't simply relay technical jargon to the client and expect her to provide useful feedback. Not only that, our team would have to create a very easy-to-use admin end to the website that she could master easily.\\

As we advanced the development of the website, we would have to be very clear and straightforward when updating her on our progress. We accomplished this mainly through emails containing concise, clear and jargon free status updates on the website progress. These updates had been sent at important stages of development such as the transition from the design inception to the early implementation. The reason for this is that we want to build something that the client wants, not something that we think the client wants. When we felt there was a need to ask the client about her opinion on some certain aspect of the website, we emailed her without hesitation and she was quite often happy to accommodate the new changes. It seemed that she was prepared to grant us some degree of trust with creating her website, which certainly helped, but we still sought her opinion of most changes that took place. Arguably then, communication with our client has been crucial to the design of this website. Our communication was put to the test when we had to get her to sign up for her own web server. We had to prepare a lengthy, detailed and broken down walk through of how to purchase website hosting. For an average user of the Internet, this task would have been far less demanding. However, our client's expertise in this domain required far more patience. \textbf{Figure 1.1} somewhat illustrates the challenge we have. It is basically the "adoption rate curve" for a percentage of users of new technologies. In other words, how quickly users are prepared to adopt a new technology after it is first released. It is known as Roger's bell curve. Our client would be at the far right end of the spectrum, in the 16\% of users described as "Laggards" in the graph.\\
	
\begin{figure}
\begin{center}
\includegraphics[scale=0.6]{DiffusionOfInnovation}
\caption{Roger's bell curve, source: http://en.wikipedia.org/wiki/File:DiffusionOfInnovation.png}
\end{center}
\end{figure}

Another perspective to consider is how users actually come to accept and use a new technology, and the factors that influence such a decision. \textbf{Figure 2.2}, otherwise known as the Technology Adoption Model (TAM) provides a graphical realisation of this. \footnote{http://en.wikipedia.org/wiki/Technology\_acceptance\_model} The model suggests that when users are presented with a new technology, a number of factors influence their decision about how and when they will use it, notably:

\begin{itemize}
\item Perceived usefulness (PU) - This was defined by Fred Davis as "the degree to which a person believes that using a particular system would enhance his or her job performance".
\item Perceived ease-of-use (PEOU) - Davis defined this as "the degree to which a person believes that using a particular system would be free from effort" (Davis 1989).	
\end{itemize}

From our early meetings with our client, we believe we can make well informed assumptions about these two aspects. Since the client has especially contacted the University to ask for such a website to be implemented, and the fact she currently has no online presence already, it is our understanding that she believes this website or new system will definitely enhance her job performance. We have some reservations about her perceived ease of use however. As is explained earlier, she is very inexperienced with Internet technologies in general and may not feel entirely comfortable when faced with the new system for the first time. It is our main task to achieve simplistic, user friendly design that she can learn to use as quickly as possible and overcome any early doubts about the system. We believe that the usefulness of the website in the nature of her work greatly outweighs the challenge of having a novice user, although it is certainly a trade off we have to be mindful of. 

\begin{figure}
\begin{center}
\includegraphics{tam}
\caption{Technology Adoption Model, source: \small{http://en.wikipedia.org/wiki/File:Technology\_Acceptance\_Model.png}}
\end{center}
\end{figure}

Reflecting on our experiences in our project, working with the client has not been exceptionally easy. One of our main challenges is that our client is a novice computer user. Quite ironically, it has been a task for us to translate regular computing jargon into layman's terms in order for them to understand. This was critical in our requirement gathering process. In spite of this additional challenge that teams working with other projects might not particularly face, it is not necessarily something that is discouraging for us. When we eventually graduate and face real world software projects, it won't always be technically minded people like ourselves we deal with, it will people who are more similar to our client. It will provide us something to drawn upon when we are asked to recount our experiences in future interviews. The opportunity to work with a real client will serve us as great preparation for working life.\\
\\
Our project lives on the Internet. Ten years ago, the Internet was everywhere, but now it's even more so. It's a rapidly evolving area where new technologies and new techniques are developed very quickly. These have been made easy to use by anyone, regardless of: computer knowledge, nature of their device (mobile or desktop) or their operating system. All they need is an up-to-date web browser - arguably an easy requirement to fulfill, since most of them update automatically. So, there is a large audience that our project can reach, and that is considerably motivating for everyone in our team who worked on this project.\\

\section{Preliminaries}
To understand this report it is necessary to understand that we do not have to implement a translation algorithm. We are providing customers with an interface to send documents to be translated by a professional translator for a fee, and then returned in the chosen translated language using the same web interface. To understand the process that we have devised it would be advantageous to have a simple understanding of how a database works. As mentioned earlier, we have adopted several frameworks in our project development. These include Bootstrap (CSS, HTML), CodeIgniter and phpMyAdmin (all of which are discussed in Section~\ref{chap:design} later). These frameworks provide advanced functionality which will, in most instances, not be necessary for our project and will not be utilised. Conversely, it allows us to demonstrate that we are professional software developers and that we are capable of software re-use.

\section{Outline}
The remainder of this report will go into more detail on the background research of our project, expand on our motivation and set out our group organisation and project plan. After this we will detail our design ideas and methodologies, before moving on to document our implementation, testing and evaluation. We will then discuss any problems encountered and the results of our evaluation before revealing the final status of the project, giving a detailed outline of the deployed site including any graphics and information relating to Bethel Translations.
 

%==============================================================================
\chapter{Design}
\label{chap:design}
\section{Requirements Gathering}


At the beginning of our project we met with our client and our project supervisor to discuss her requirements. Our client currently translates for an agency, on an ad-hoc basis, and is looking to create her own translation business. To this extent she wishes to have a website built to allow her to gain an online presence in translation and to digitize the process of receiving and sending translations. We were armed with a long list of questions and ideas for the meeting and a transcript is included as an appendix.\\
\\
As software developers naturally would when creating anything from scratch, our team looked for similar websites already in existence. We identified common useful features of each, and also those features that were not so useful, and listed some we thought could be useful but simply did not exist in any of the sites we examined. One recurring theme we noticed in a majority of similar translation websites was that the home page was very cluttered and full of text. In other words, the process that the user had to follow to obtain some translation of a document was not extremely clear. Instead, they were met with various registration options, other services and annoying advertisements. \textbf{Figure 2.1} is a prime example of such bad design practices. It is cluttered, unclear and unattractive to potential customers. The most intruiging thing about the website in particular is that is ranked \textit{first} in a Google search for the term "document translation". Rather alarmingly it seems this site has prioritised search engine optimisation over actual usability, or they have spent most of their budget paying to be rated first. Whilst being rated first is an advantage to the amount of business received, it should not detract from giving the user a usable and enjoyable experience. We therefore felt encouraged that there was certainly room on the online domain for a document translation service that was simpler, better designed and more intuitive to use.

\begin{figure}
\begin{center}
\includegraphics[scale=0.4]{ex1doctrans}
\caption{Online document translator, source: http://www.onlinedoctranslator.com/}
\end{center}
\end{figure}

From previous modules in our degree, namely IM2 and IS3, our team had experience of applying Jakob Nielsen's heuristics to obtain a successful user interface. We wanted to develop the idea that our website would display the minimal amount of information to a user by providing the registration, document upload and language selection elements on a single page, in a simple 3 step process. Our interface would then fall in line with the principle that user interfaces should have an aesthetic and minimalist design. \footnote{http://www.useit.com/papers/heuristic/heuristic\_list.html} We believe this is an extremely important aspect of any modern website based on the way that users make a decision of whether or not to use the services offered by the website. For example, imagine a user enters a Google search for ``Translation service'' and clicks our website in the results page. If the page they are met with looks too complicated or confusing in nature, the user simply clicks ``Back'' on their web browser, and goes to the next appropriate web page. If, however, the website looks clean, simple, and easy to use, the user would be more inclined to use it properly. We believe that our 3-step process found on the home page encourages anyone that requires a translation service for the provided languages to at least try for a quote, if not go through with the whole process.\\
\\
With this approach in mind, we began to create some wireframes in order to form a solid idea of how this 3-stage process would physically look on our website. \textbf{Figure 2.2} illustrates our early attempt at doing so.

\begin{figure}
\begin{center}
\includegraphics[width=\linewidth]{wireframes/bt-3step}
\caption{The 3-step process}
\end{center}
\end{figure}

The layout is intuitive, flowing and simple to approach. In this early design we have split the page into two sections. In section 1 the user is presented with a short amount of precise text detailing the main function of the site. Subsequently, in Section 2 the user simply enters some details, adds the documents, sets the requirements and clicks a button to request a quote. \\There is no daunting, large form based entry that some other websites encapsulate. It is minimal and it is undemanding. Obviously, the majority of the work done by our system would be when the user clicks ``Get your quote'', and the system would have to register a user (pending email validation), upload their documents to the file store, associate a \textit{jobid} and requirements, and place the job in our client's pending work queue. The implementation of such functions are discussed later in Section~\ref{chap:impl}.\\
\\


\section{User Process}
With this 3 stage process as our main design factor of the website, we began to consider the main user groups of the website. We identified the needs of two main user classes: the \textbf{customers} and the \textbf{translator}

The clients are the users of the service, the visitors of the website. However, as far as the system is concerned, not all visitors are clients, because, in order for a visitor to become a client, he must register with the service. So we decided to have a visitor as a category with a registered user a subcategory. A registered user would then possess all the same abilities as a visitor with some specialised capabilities:

\textbf{Visitor}
\begin{enumerate}
\item{Rationale: The visitor is just browsing. An anonymous visitor of the website.}
\item{Background: "I need to get some documents translated. I came across this website and before I register or send any of my documents, I want to make sure that I'm dealing with a serious service."}
\end{enumerate}
\begin{itemize}
\item{He/She is a potential client, therefore the steps which he must make in order to become one must be as clear as possible.}
\item{His/Her \textbf{goal} is to inform himself/herself about the service. In order for him/her to transition to being a customer, he/she must be convinced that the service provided is of great quality, so the system's goal is to make itself trustworthy. Also, a clear privacy policy regarding e-mail addresses and the documents should be available, since they might contain sensitive data.}
\end{itemize}
\textbf{Customer}
\begin{itemize}
\item{Rationale: The customer is a registered user of the system.}
\item{He/She has the same goals as the visitor, but, now that he/she is registered, he/she trusts the service a bit more. He/She has access to all the documents that he/she ever submitted for translation and can view each of the \textbf{job statuses} for which his/her documents are contained within. He/She can view documents he/she has paid for, and up to a certain period of time, download both the original copy and the translated copy. He/She can also view some other useful statistics on his/her previous jobs.}
\end{itemize}
\textbf{Administrator/Translator}
\begin{itemize}
\item{Rationale: The translator is the one answering all the translation requests.}
\item{"As a translator, I must review the documents that my clients send me and quote them. If they accept the quote, I must also translate them and let them know when the work is complete."}
\item{His/Her goal is to answer all of the client's requests.}
\end{itemize}

\begin{figure}[h]
\centering
\includegraphics[width=\linewidth]{jobstatuses}
\caption{The transition of jobs through "statuses"}
\end{figure}

Building upon the needs of these user classes, we needed to design a flexible system that provided functionality for all of these goals. Our main focal point would be the transition of \textbf{jobs}. In the context of our system, a job is what the website creates when the user uploads one or more documents for translation. Clients (registered users) would submit and pay for them. The translator would review them, download them and upload them. We considered the transition of jobs throughout the life cycle of a translation, as job statuses, and produced a sequence that is illustrated in \textbf{Figure 2.3}. \newline

The diagram is much self-explanatory, but to summarise: Jobs that are submitted by the clients are placed in a \textbf{pending} work queue. The translator is able to review these documents and send some quote to the client. After a job has been quoted, it is placed in the \textbf{Quotes} queue. Quoted jobs are held in this queue until they are paid for via Paypal. When this has happened, they are moved to \textbf{Translations}, which is a breakdown of completed jobs. After a period of 90 days, jobs from this section are moved to \textbf{History} in order to reduce space on the server.

The transformation from this design plan to a viable user interface that is built upon these job statuses is described later in Section~\ref{chap:ui}.
\newline

\newpage
\begin{wrapfigure}{l}{0.42\textwidth}
\vspace{-20pt}
  \begin{center}
    \includegraphics[width=0.4\textwidth]{userProcessClient.png}
  \end{center}
  \caption{userProcessClient}
\vspace{-60pt}
\end{wrapfigure}

First the client will have the option to switch among 3 displaying languages, namely English, French and Italian. \\

To use the translation service, one must first register by: inputting name and e-mail address, uploading one or more file(s), and setting translation requirements (source/desire language, currency and due date). Note that uploading is required so that only those who are actually going to use the service are allowed to register. Also, e-mail activation is required to finish the registration. If the system finds that the inputted E-mail address has been registered, the client will be asked to login.\\

Either by finishing the registration process or by submitting a new job (uploading file(s) and setting requirements) after login, a new job is submitted.\\

Meanwhile the translator will quote the job and a negotiation on details such as price or due date will possibly happen during this stage (it is optional, however).\\

If the client decides to accept the quote she/he will need to pay immediately through PayPal or could just leave it as quoted. Then the file(s) will (hopefully) be translated by the due date and a download link of the translation(s) will be offered in the client dashboard.\\

If the client decides to decline the offer, the translation service process will be canceled.\\

Other than those above, the client may: \\

	Navigate the links on the main menu: read the “About” page (information about the service and the translator); read the “Testimonials” page; contact the translator in the “Contact” page through the contact form; request other services (Edit and Proofreading, Over-the-phone Interpreting and Video Remote Interpreting) in corresponding “Service” page through the contact form. \\


\newpage

\begin{wrapfigure}{l}{0.36\textwidth}
\vspace{-20pt}
  \begin{center}
    \includegraphics[width=0.34\textwidth]{userProcessTranslator.png}
  \end{center}
  \caption{userProcessTranslator}
\vspace{-60pt}
\end{wrapfigure}
First the translator will be directed to the admin dashboard, in which she/he will (probably):\\

Check all pending jobs, and pick one to quote.\\ 

The quoting process will include: check the job requirements, download the files uploaded in the job, read through the documents, set a quote and confirm it in the dashboard. During this process, the translator may reject a job within any stage.\\

A negotiation between the translator and a client may happen during or after the quoting stage. They may communicate via e-mails.\\

If no rejection or declination happens, the translator will then be waiting for the client to pay. Only after the payment is successful an actual translation process will start.\\

Finally the translator will upload the translated documents to the server through the admin dashboard.\\

Other than those above, the translator may: \\

Check processed jobs with different status in “Quoted”, “Accepted”, and “Declined” sections in the admin dashboard respectively. Also all recent translations can be downloaded through “Translations” and all past translation records can be found in the “History” section.\\

View the statistic data for the website in “Site Statistics”.\\

	Navigate the links on the main menu: read the “About” page (information about the service and the translator); read the “Testimonials” page.\\


\newpage
\rule{430pt}{1pt}


% \newpage
\section{System Design and Wireframes}
After laying out the process of the website we focused on organizing the content
and on the wireframes.

\subsection{Information Architecture}
We analyzed the requirements and made a website structure diagram
(Figure ~\ref{website-structure}).

\begin{figure}[h]
\centering
%trim top bottom
\includegraphics[height=200px, trim=0px 450px 0px 100px]{website-structure}
\caption{Website structure}
\label{website-structure}
\end{figure}

\newpage
\subsection{Wireframes}
% 
% paper prototypes, then functional wireframes
% http://webstyleguide.com/wsg3/2-universal-usability/5-in-design-process.html
% 
Wireframes are blueprints of the website, representing every important page and 
their structure and behaviour.
They focus more on what elements would a page contain, and their place on
the page, rather than a refined design. Therefore, these sketches take form of
rough drawings of the final design, usually keeping things like images or
colours out, because they can distract the reader from his main purpose, which
is analyzing the layout of that particular page, rather than the visual design.
% some more on why these are useful, etc.


Conventions used in the wireframes:
\begin{itemize} \itemsep1pt \parskip0pt \parsep0pt
	\item \textbf{Green background} and/or \textbf{yellow labels} - are used to
	indicate specific groups of content that will be referenced in the document.
	\item \textbf{links} - blue text and underlined
\end{itemize}

By the nature of their content, the pages can be categorised into static and
dynamic. Static pages display the same information for all users, regardless of
their status (logged in or just visiting, client or administrator). About,
Testimonials and Contact are such pages in the system described here. On the
other hand, dynamic pages draw some data from the database and display it
accordingly. In this case, the dynamic pages are the ones in the dashboard
(both client’s and administrator’s) since the information there is specific to
every client.



\subsubsection*{Template}
All the pages have this basic layout comprised of the following sections:
\begin{itemize} \itemsep1pt \parskip0pt \parsep0pt
	\item Header
	\item Content area
	\item Footer
\end{itemize}

\begin{figure}
\centering
%trim left bottom
\includegraphics[width=\linewidth, trim = 0px 50px 0px 50px]
	{wireframes/main-layout}
\caption{Main layout}
\label{wireframes-main-layout}
\end{figure}

The \textbf{header} and the \textbf{footer} sections will remain unchanged
across all pages and the rest of the content goes in the \textbf{content area}.
This is a good design practice because it keeps the entire website consistent. In 
terms of technological benefits this will increase the speed of loading the website 
due to lower bandwitdh requirements. The browser does not need to reload the header 
and footer every time, as it only loads the new sections. The browser can cache
these sections also meaning that the site is more efficient.
The rest of the sketches of the pages illustrated here will only show the
content area.

The header contains:
\begin{itemize} \itemsep1pt \parskip0pt \parsep0pt
	\item \textbf{the name and/or logo of the website.} This is key as it promotes the translators brand. Bethel Translations should be prominent to all clients so that they know what website they are on, they can use this to recall the services it provides and also recognise the quality of the brand at the same time. 
	\item main navigation menu
\end{itemize}

The menu is a collection of links to the main areas of the website:
\begin{itemize} \itemsep1pt \parskip0pt \parsep0pt
	\item Home
	\item About
	\item Services
	\item Testimonials
	\item Contact
	\item Login - this link must stand out, since it is of a higher
	importance and the page linked is of a different nature than the rest,
	so it is separated from the others.
\end{itemize}


Footer:
\begin{itemize} \itemsep1pt \parskip0pt \parsep0pt
	\item copyright notice, if needed
	\item links to legal documents (e.g.: Privacy policy, terms of use, etc)
	\item awards or certifications (translation services related, 
	PayPal certification)
	\item contact information (actual information or just a link to the 
	`Contact’ page)
	\item link to the business’ facebook page
\end{itemize}
% end of Template


\subsubsection{Main page}
From our earlier decision of creating a simple three step process and
implementing this on our homepage, we took our wireframe and refined it to
come up with the  \textbf{Main Template} (Figure ~\ref{wireframes-main-layout}).

\begin{figure}[ht]
\label{wireframes:main-template}
\begin{center}
\includegraphics[width=\linewidth]{wireframes/bt-homepagev3}
\caption{The Main Template}
\end{center}
\end{figure}

The site-wide header (labelled 1) and footer (labelled 3) can clearly be seen to be simple and uncluttered.
% end Main page


\subsubsection{About}
This is the page the users would go to in order to find out more about the
service. Contains a few paragraphs that detail goals and accomplishments.
The page needs to answer some possible questions that the users might have
regarding the business:
\begin{itemize}
	\item who is behind it?
	\item what are they doing?
	\item when did they start doing it?
	\item where are they?
	\item how are they accomplishing what they claim to do?
	\item Optionally an image, or even a short video, to enhance trustworthiness.
\end{itemize}

\begin{figure}
\label{wireframes:about}
\begin{center}
\includegraphics[width=\linewidth, trim = 0px 110px 0px 100px]{wireframes/about}
\caption{The About page}
\end{center}
\end{figure}
% end About


\subsubsection{Testimonials}
Having testimonials from happy customers also adds to the trust of the business.
The business owner would ask her clients for feedback and permission to publish
it on the website. Then she can pick which ones would suit her and post only a
fragment on the website, along with some details of that client (name, company,
occupation).

\begin{figure}
\label{wireframes:testimonials}
\begin{center}
\includegraphics[width=\linewidth, trim = 0px 80px 0px 220px]
	{wireframes/testimonials}
\caption{The Testimonials page}
\end{center}
\end{figure}
% end Testimonials


\subsubsection{Contact}
Having other contact methods (phone number, physical mailing address) also add
to the credibility of the business. ``A company with no address is not one you
want to give money to.'' [Jakob Nielsen]
\begin{enumerate} \itemsep1pt \parskip0pt \parsep0pt
	\item work telephone number
	\item physical address
	\item social media pages
	\item contact form (by e-mail)
\end{enumerate}

\begin{figure}
\label{wireframes:contact}
\begin{center}
\includegraphics[width=\linewidth, trim = 0px 40px 0px 220px]
	{wireframes/contact}
\caption{The Contact page}
\end{center}
\end{figure}
% end Testimonials

\subsection{Summary}
From the images and diagrams in this chapter we have constructed the blueprints for a modern, stylish, and most significantly - user friendly website. Ultimately the website we want to build is one the client can fully use, without any reluctancy. The next chapter, \textbf{Implementation}, describes how transitioned from design to implementation, and the methods used in doing so.

%==============================================================================
\chapter{Implementation}
\label{chap:impl}

In this chapter, we describe how we implemented the system from our design plan and detail the technologies used in doing so.


%------------------------------------------------------------------------------

\section{Development Environment}
Programming for the project was split between several web-orientated languages. The central development language was PHP for developing the controllers and models. HTML5, JavaScript and jQuery were used in the views. As mentioned previously we employed the use of two frameworks, namely CodeIgniter and TankAuth. CodeIgniter provided the Model-View-Controller architecture PHP framework for the website, and TankAuth is an open source authentication library for CodeIgniter.\newline
\newline
As we learned from the Distributed Information Management course we studied this year, web-development is ever-changing and to stay current developers must adapt to using new technology. As a result of this we have coded the views using the newest HTML5 standard. This will allow future developers to maintain the site with ease as adding new modern features will be simpler.\newline
\newline
Another thing we learnt from our Distributed Information Management course is that it is desirable to separate out our concerns when developing in a web environment. This led us to using the open source Twitter Bootstrap \url{http://twitter.github.com/bootstrap/}. Twitter Bootstrap is “simple and flexible HTML, CSS, and Javascript for popular user interface components and interactions”. During implementation Twitter released an upgraded version, version 2.0, of Bootstrap and we upgraded to this version when it was released in early 2012.\newline
\newline
For implementation purposes we set up an online SVN using Google Code. This version control system allowed us to keep track of changes, report and resolve issues, and maintain a wiki of useful pieces of information that we needed to keep and track.\newline
\newline
Further to this we also set up a test server to allow us to view our site live on the web as we developed it and to test any changes as we made them.\newline
\newline


% - - - - - - - - - - - - - - - - - - - - - - - - - - - - - - - - - - - - - - -
%%\subsection{TBC - think project specific chapter}

%%TBC

%------------------------------------------------------------------------------


\section{Database Model}

Our database structure was designed after we had thoroughly revised our user registration and job transition processes. We envisaged there being four separate entities: one for \textbf{customers} to become registered, one to represent \textbf{documents} being submitted, another for the \textbf{jobs} that are comprised of submitted documents and finally one for the documents once they have been \textbf{translated}. The attributes of each of these entities and the relationships between them is illustrated in \textbf{Figure 3.1} 

\begin{figure}
\begin{center}
\includegraphics[width=\linewidth]{bt-dbstruct}
\caption{ER Diagram for Bethel Translations}
\end{center}
\end{figure}

The majority of activity for these tables occurs in the first three entities, as they are all populated in some way during the 3 stage process discussed earlier.

\section{Database Schema}
TBC

\section{Prototype}
TBC

\section{User Interface}
TBC

%==============================================================================
\chapter{Evaluation}

First feedback with Karen incl wireframes etc - Jan 25\\
Second feedback session with Karen incl more complete website design - Feb 9th

\section{Testing}

\section{Translator Evaluation}

\section{Client Evaluation}

%==============================================================================
\chapter{Conclusion}

Team photo goes here? :)
A great project! etc..

%==============================================================================
\section{Aims}
TBC

\section{Achievements}
TBC

\section{Future Work}
TBC

\section{Contributions}

Alasdair done this...
Andrei handled that...
Stephen took responsibility for... 
Paul mainly done...
Wei was responsible for...	

%==============================================================================


%==============================================================================
\bibliographystyle{plain}
\bibliography{example}
\appendix
\chapter{Glossary of Terms}
\label{sec:gloss}

\begin{itemize}
\item{\textbf{Free-lance} - Working for different companies at different times rather than being permanently employed by one company.} 
\item{\textbf{Client} - A person or organization using the services of a professional person or company.} 
\item{\textbf{Users} - A set of people who use or operate something, esp. a computer or other machine.}
\item{\textbf{Requirement gathering} - Determining the needs of a client through any form of communication.}
\item{\textbf{Software project} - Using the surrounding context, a software project aims to create application(s) using programming language(s) by adhering to project management principles.}
\item{\textbf{Programming language} - A programming language is an artificial language designed to express computations that can be performed by a computer.}
\item{\textbf{Web scripting language (\textit{PHP, Javascript})} - A scripting language is a programming language that allows control of one or more applications.}
\item{\textbf{Website development} - The process of constructing and maintaining a website.}
\item{\textbf{LAMP} - LAMP, (Linux, Apache, MySQL and PHP), is an acronym for a solution stack of free, open source software}
\item{\textbf{Web application framework} - A software framework that is designed to support the development of dynamic websites }
\item{\textbf{Open source} - Computer software for which the code is freely available }
\end{itemize}
%==============================================================================
\chapter{Project Plan}

\section{Introduction}
\subsection{Identification}
\subsection{Related Documentation}
\subsection{Purpose and Description of Document}
\subsection{Document Status and Schedule}

\section{Resources, Budgets, Schedules and Organisation}
\subsection{Work Breakdown Structure} 
\subsection{Resource Estimation and Allocation to WBS}
\subsection{Schedules}
\subsection{Equipment, Materials, Facilities, and Other Resources}

\section{Assurance Plan}                                                                               

\section{Risk Management Plan}
\subsection{Risk Identification and Analysis} 
\subsection{Monitoring}
\subsection{Avoidance}
\subsection{Mitigation}
\subsection{Review}
\subsection{List of Managed Risks}

\section{Configuration Management Plan}



%==============================================================================
\chapter{Evaluation Documents}
\section{Introduction and Consent}
\textbf{Bethel Translations – Evaluation Consent Form}\newline
This evaluation will take about 20 minutes to complete. 
You may ask as many questions as you like before the evaluation starts. The tasks you have to carry out will be provided to you on another sheet. You are asked to circle YES if you successfully complete a task or NO if the opposite.\newline
When uploading files please do not upload private or assessed documents – lecture PDFs are an example of a good document to upload. All documents will be deleted after the evaluation and will not be opened or parsed.\newline
When you have completed all tasks please tell the person in charge of the evaluation and they will direct you to the online questionnaire that has to be filled out to complete the evaluation.
All results will be held in strict confidence, ensuring the privacy of all participants. No personal participant information will be stored with the data. Online data will be stored in a password protected computer account; paper data will be kept anonymous. 
 Your participation in this experiment will have no effect on your marks for any subject at this, or any other university. \newline
Please note that it is the website, not you, that is being evaluated. You may withdraw from the experiment at any time without prejudice, and any data already recorded will be discarded. \newline If you have any further questions regarding this experiment, please contact: \newline
Team O:  teamo@stbernadettes.co.uk \newline
		or \newline
Karen Renaud (Team Supervisor): Karen.Renaud@glasgow.ac.uk\newline
\rule{430pt}{1pt}

I have read this information sheet, and agree to voluntarily take part in this experiment: 
 
Name: \rule{200pt}{1pt} \newline \newline
Email: \rule{200pt}{1pt} \newline \newline
Signature: \rule{180pt}{1pt} \newline \newline
Date: \rule{100pt}{1pt} 
Age:  \rule{100pt}{1pt} \newline

 \section{Task Sheet}
\textbf{Task Sheet: Client}\newline 
Please navigate to www.betheltranslations.com then complete the tasks below. \newline \newline
\textbf{Task 0}: Navigate to all the pages of the site and write down the first thoughts you have about the site. Please write on the other side of this page. \newline \newline
\textbf{Task 1}: Please fill out the form on the home page and upload a doc. Please choose English to French translation. \newline \newline
Successful?    YES      NO \newline \newline
\textbf{Task 2}: Log in to your dashboard.  \newline \newline
Successful?    YES      NO \newline \newline
\textbf{Task 3}: Check the status of your document and submit one more (French to Italian) Then accept one quote, \newline \newline
Successful?    YES      NO \newline \newline
\textbf{Task 4}: Download your translations \newline \newline
Successful?    YES      NO \newline \newline
\textbf{Task 5}: Contact Joelle to negotiate the price \newline \newline
Successful?    YES      NO \newline \newline
\textbf{Task 6}: Logout and visit the facebook page to leave some positive feedback. \newline \newline
Successful?    YES      NO \newline \newline
 
 
    


\end{document}
