\documentclass{article}
\title{An Introduction to the Translation Website Project of Team O}
\author{Stephen Hayton}
\date{15th January 2012}

\begin{document}

\maketitle

\textit{\small{This article assumes a basic general computing knowledge from the reader. If any words or phrases are not understood, consult the glossary of terms at the end of the document.}}\\
\\
\\
As part of our degree, in year 3 we are tasked with a team project, relating to computing disciplines new to us and drawing on knowledge gained from the preceding two years at University. Teams were randomly assembled at the beginning of semester and my team received the project to create a website containing a translation service for a real client. I was personally pleased to get this allocation mainly due to the latter part of the task: the fact we would be working with a real client. Before I delve into details about our client however, here is a brief summary of our project at hand: We are to develop a website for a free-lance translator which allows users to upload documents and have them eventually translated by our client, after the user has accepted quotes and paid for the translations of the documents.\\ 
\\
Reflecting on our experiences in our project to date, working with the client has not been exceptionally easy. One of our main challenges is that our client is a novice computer user. Quite ironically, it has been a task for us to translate regular computing jargon into layman's terms in order for them to understand. This was critical in our requirement gathering process, which is still ongoing. In spite of this additional challenge that teams working with other projects might not particularly face, it is not necessarily something that is discouraging for myself. When we eventually graduate and face real world software projects, it won't always be technically minded people like ourselves we deal with, it will people who are more similar to our client. It will provide us something to drawn upon when we are asked to recount our experiences in future interviews.  I believe that the opportunity to work with a real client will serve us as great preparation for working life.\\
\\
An additional benefit of our project is the knowledge I will develop in programming languages while creating the website. I would have considered myself to have no prior knowledge in web scripting languages such as PHP and Javascript before the project began. I feel much more comfortable using them now. These are popular languages that are sought out by employers in the Computing industry, so it is reassuring to know I am learning valuable website development skills as the project progresses.\\
\\
There are multiple websites on the Internet that offer a similar service to the one we have to develop. This creates a desire of competing with what's already available by trying to improve areas where other websites have fallen short. We believe one of the key factors that makes a modern website successful is minimalistic design: it offers simple and effective functionality and is aesthetically pleasing. The combination of these things means users are more likely to use the website after stumbling across it in a search, perhaps, and will ultimately give our client a larger customer base. To clarify, we are not trying to re-invent the wheel with our project. We will and have so far used frameworks and other free source components in the development of our website. We are trying to maintain a user friendly feel and look and believe that we are on the right path with that. One of the things that encourages this notion is the fact we have created a simple 3-stage process in which users can register, upload documents, and request languages to translate to.\\   
\\
I feel that our team will embrace all of the challenges that our project entails. I have so far developed some excellent working relationships with other members of the team whom I did not even know before last September. The vast majority of software engineering employers will expect new recruits to work efficiently in teams full of new people, so this aspect of our project is yet another experience that will prove of great assistance in the future.
\section{Glossary of terms}

\begin{itemize}
\item{\textbf{Free-lance} - Working for different companies at different times rather than being permanently employed by one company.} 
\item{\textbf{Client} - A person or organization using the services of a professional person or company.} 
\item{\textbf{Users} - A set of people who use or operate something, esp. a computer or other machine.}
\item{\textbf{Requirement gathering} - Determining the needs of a client through any form of communication.}
\item{\textbf{Software project} - Using the surrounding context, a software project aims to create application(s) using programming language(s) by adhering to project management principles.}
\item{\textbf{Programming language} - A programming language is an artificial language designed to express computations that can be performed by a computer.}
\item{\textbf{Web scripting language (\textit{PHP, Javascript})} - A scripting language is a programming language that allows control of one or more applications.}
\item{\textbf{Website development} - The process of constructing and maintaining a website.}
\end{itemize}
\end{document}
